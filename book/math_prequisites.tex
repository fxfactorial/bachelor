\chapter{Mathematical prequisites \label{chap:preq}}

In this chapter we formalize the notation used in the rest of the thesis, and
furthermore attempt to discuss and study the elementary functions on which the
project has been grounded.
\\
The $\ll$ and $\gg$ are respectively used with the meaning of left and right
bitwise shift, as usual in computer science.
\\
The $\dsqrt$ function will be defined in section \ref{sec:preq:sqrt}, with the
acceptation of discrete square root.
\\
The logarithmic $\log$ function is assumed to be in base two, i.e. $\log_2$.
\\
The $\idiv$ symbol is the integer division over $\naturalN$, i.e.
$a \idiv b = \floor{\frac{a}{b}}$.

\section{Algorithmic Complexity Notation}
The notation used to describe asymptotic complexity follows the $O$-notation,
abused under the conventions and limits of MIT's Introduction to Algorithms.

Let \bigO{g} be the asymptotic upper bound of g:
$$
\bigO{g(n)} = \{ f(n) : \exists n_0, c \in \naturalN \mid 0 \leq f(n) \leq cg(n)
\ \forall n > n_0 \}
$$

With $f(n) = \bigO{g(n)}$ we actually mean
$f(n) \in \bigO{g(n)}$.

\section{Euclid's Greatest Common Divisor}

Being the greatest common divisor a foundamental algebraic operation in the TLS
protocol, \openssl implemented it with the following signature:

\begin{minted}[fontsize=\small]{c}
  int BN_gcd(BIGNUM *r, BIGNUM *a, BIGNUM *b, BN_CTX *ctx);
\end{minted}

The computation proceeds under the well-known Euclidean algorithm, specifically
the binary variant developed by Josef Stein in 1961 \cite{AOCPv2}. This variant
exploits some interesting properties of $gcd(a, b)$:
\begin{itemize}
  \setlength{\itemsep}{1pt}
  \setlength{\parskip}{0pt}
  \setlength{\parsep}{0pt}
  \item if $a,\ b$ are even, then $gcd(a, b) = 2gcd(a/2, b/2)$;
  \item if $a$ is even and $b$ is odd, then $gcd(a, b) = gcd(a/2, b)$;
  \item  $gcd(a, b) = gcd(a-b, b)$, as in the standard Euclid algorithm;
  \item the sum of two odd numbers is always even.
\end{itemize}

% Donald Knuth, TAOCP, "a binary method", p. 388 VOL 2
Both \cite{AOCPv2} and \cite{MITalg} analyze the running time of the
algorithm; \cite{MITalg}'s proof is fairly simpler and proceeds %elegantly
by induction.
Anyway, both show that algorithm ~\ref{alg:gcd} belongs to the class
\bigO{\log b}.

\begin{algorithm}[H]
  \caption{\openssl's GCD \label{alg:gcd}}
  \begin{algorithmic}[1]
    \State $k \gets 0$
    \While{$b \neq 0$}
      \If{$a$ is odd}
        \If{$b$ is odd}
          \State $a \gets (a-b) \gg 1$
        \Else
          \State $b \gets b \gg 1$
        \EndIf
        \If{$a < b$} $a, b \gets b, a$ \EndIf

      \Else
        \If{$b$ is odd}
          \State $a = a \gg 1$
          \If{$a < b$} $a, b = b, a$ \EndIf
        \Else
          \State $k \gets k+1$
          \State $a, b \gets a \gg 1, b \gg 1$
        \EndIf
      \EndIf
    \EndWhile
    \State \Return $a \ll k$

  \end{algorithmic}
\end{algorithm}

Unfortunately, there is yet no known parallel solution that significantly improves
Euclid's \textsc{gcd}.


\section{Square Root \label{sec:preq:sqrt}}

Computing the square root is another important building block of the project,
though not available in \openssl\!.
Apparently,
% \openssl is a great pile of crap, as phk states
\openssl does only provide the discrete square root implementation using the
Tonelli/Shanks algorithm, which specifically solves in $x$ the equation
$x^2 = a \pmod{p}$, with $p \in \naturalPrime$:

\begin{minted}{c}
  BIGNUM* BN_mod_sqrt(BIGNUM* x, const BIGNUM* a, const BIGNUM* p,
                      const BN_CTX* ctx);
\end{minted}

Instead, we are interested in finding the the pair
$\angular{x, r} \in \naturalN^2 \mid x^2 + r = n$, that is, the integer part of
the square root of a natural number and its rest.
Hence, we did come out with our specific implementation, first using Bombelli's
algorithm, and later with the one of Dijkstra. Both are going to be discussed
below.

Unless otherwise specified, in the later pages we use $\sqrt{n}$ with the
usual meaning ``the half power of $n$'', while with $x, r = \dsqrt{n}$ we mean
the pair just defined.

\paragraph{Bombelli's Algorithm \label{par:preq:sqrt:bombelli}} dates back to
the XVI century, and approaches the problem of finding the square root by using
continued fractions. Unfortunately, we weren't able to fully assert the
correctness of the algorithm, since the original document
~\cite{bombelli:algebra} presents a difficult, inconvenient notation. Though,
for completeness' sake, we report in table
~\ref{alg:sqrt:bombelli} the pseudocode adopted and tested for its correctness.

\begin{algorithm}[H]
  \caption{Square Root: Bombelli's algorithm}
  \label{alg:sqrt:bombelli}
  \begin{algorithmic}[1]
    \Procedure{sqrt}{$n$}

    \State $i, g \gets 0, \{\}$
    \While{$n > 0$}
      \State $g_i \gets n \pmod{100}$
      \State $n \gets n // 100$
      \State $i++$
    \EndWhile

    \State $x, r \gets 0, 0$
    \For{$j \in \;  [i-1..0]$}
      \State $r = 100r + g_i$
      \For{$d \in \; [0, 9]$}
        \State $y' \gets d(20x + d)$
        \If{$y' > r$}  \textbf{break}
        \Else  \ \ $y \gets y'$
        \EndIf
      \EndFor
      \State $r \gets r - y$
      \State $x \gets 10x + d - 1$
    \EndFor

    \State \Return $x, r$

    \EndProcedure
  \end{algorithmic}
\end{algorithm}

For each digit of the result, we perform a subtraction, and a limited number of
multiplications. This means that the complexity of this solutions belongs to
\bigO{\log n \log n} = \bigO{\log^2 n}
\paragraph{Dijkstra's Algorithm \label{par:preq:sqrt:dijkstra}} can be found in
\cite{Dijkstra:adop}, \S 8, p.61. There, Dijkstra presents an elightning
process for the computation of the square root, making only use of binary shift
and algebraic additions.
Specifically, the problem attempts to find, given a natual $n$, the integer $a$
that establishes:
\begin{align}
  \label{eq:preq:dijkstra_problem}
  a^2 \leq n \: \land \: (a+1)^2 > n
\end{align}

Take now the pair $\angular{a=0, b=n+1}$, and consider the inverval
$[a, b[$. We would like to reduce the distance between the upper bound $b$ and
the lower bound $a$, while holding the guard \ref{eq:preq:dijkstra_problem}:

\begin{align*}
  a^2 \leq n \: \land \: b > n
\end{align*}

%% XXX. I am not so sure about this, pure fantasy.
The speed of convergence is determined by the choice of the distance $d$, which
analougously to the dicotomic search problem, is optimal when
$d = (b-a) \idiv 2$.

\begin{algorithm}[H]
  \caption{Square Root: an intuitive, na\"ive implementation}
  \label{alg:sqrt:dijkstra_naif}
  \begin{algorithmic}[1]
    \State $a, b \gets 0, n+1$
    \While{$a+1 \neq b$}
      \State $d \gets (b-a) \idiv 2$
      \If{$(a+d)^2 \leq n$}
         $a \gets a+d$
      \ElsIf{$(b-d)^2 > n$}
         $b \gets b-d$
      \EndIf
    \EndWhile
    \State \Return a
  \end{algorithmic}
\end{algorithm}

% heh, there's not much to explain here, that's almost the same in Dijkstra's
% book, excluding the inspirative familiar portrait that led to the insight of
% this change of varaibles.
Now optimization proceeds with the following change of variables:
$c = b-a$,
$p = ac$,
$q = c^2$,
$r = n-a^2$;
For any further details and explainations, the reference is still
\cite{Dijkstra:adop}.

\begin{algorithm}[H]
  \caption{Square Root: final version}
  \label{alg:sqrt:dijkstra}
  \begin{algorithmic}
    \State $p, q, r \gets 0, 1, n$
    \While{$q \leq n$} $q \gets q \gg 2$ \EndWhile
    \While{$q \neq 1$}
      \State $q \gets q \ll 2$
      \State $h \gets p+q$
      \State $p \gets q \ll 1$
      \State $h \gets 2p + q$
      \If{$r \geq h$} $p, r \gets p+q, r-h$ \EndIf
    \EndWhile
    \State \Return p
  \end{algorithmic}
\end{algorithm}

A fair approximation of the magnitude of the Dijkstra algorithm can be studied
by looking at the pseudocode in ~\ref{alg:sqrt:dijkstra_naif}. Exactly as in
the dicotomic search case, we split the interval $[a, b]$ in half on each step,
and choose whether to take the leftmost or the rightmost part. This results in
$log(n+1)$ steps. During each iteration, instead, as we have seen in
~\ref{alg:sqrt:dijkstra} we just apply summations and binary shifts, which are
upper bounded by \bigO{\log{n}/2}. Thus, the order of magnitude belongs to
\bigO{\log^2{n}}.

\paragraph{}
Even if both algorithms presented have \emph{asymptotically} the same
complexity, we believe that adopting the one of Dijkstra has lead to a
pragmatic, substantial performance improvement.


\section{RSA Cipher \label{sec:preq:rsa}}

The RSA cryptosystem, invented by Ron Rivest, Adi Shamir, and Len Adleman
~\cite{rsa}, was first published in August 1977's issue of
\emph{Scientific American}. In its basic version, this \emph{asymmetric} cipher
works as follows:
\begin{itemize}
  \item choose a pair $\angular{p, q}$ of \emph{random} \emph{prime} numbers;
    let $N$ be the product of the two, $N=pq$, and call it \emph{public modulus};
  \item choose a pair $\angular{e, d}$ of \emph{random} numbers, both in
    $\integerZ^*_{\varphi(N)}$, such that one is the multiplicative inverse of the
    other, $ed \equiv 1 \pmod{\varphi(N)}$ and $\varphi(N)$ is Euler's totient
    function;
\end{itemize}
Now, call $\angular{N, e}$ \emph{public key}, and $\angular{N, d}$
\emph{private key}, and let the encryption function $E(m)$ be the $e$-th power of
the message $m$:
\begin{align}
  \label{eq:rsa:encrypt}
  E(m) = m^e \pmod{N}
\end{align}
while the decryption function $D(c)$ is the $d$-th power of the ciphertext $c$:
\begin{align}
  \label{eq:rsa:decrypt}
  D(c) = c^d \equiv E(m)^d \equiv m^{ed} \equiv m \pmod{N}
\end{align}
that, due to Fermat's little theorem, is the inverse of $E$.

\paragraph{}
%% less unless <https://www.youtube.com/watch?v=XnbnuY7Kxhc>
From now on, unless otherwise specified, the variable $N=pq$ will always refer
to the public modulus of a generic RSA keypair, with
$p, q$ being the two primes factorizing it, such that $p > q$.
 Again, $e, d$ will respectively refer to the public
exponent and the private exponent.

%%% Local Variables:
%%% mode: latex
%%% TeX-master: "question_authority"
%%% End:
