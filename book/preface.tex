\chapter{Preface}

Even if the basic RSA key generation algorithm is fairly straightforward, it
turns out that any software willing to provide such a feature does have to test
the candidate key against a substantious number of threats before claiming its
security.

The purpose of this project is to examine the TLS protocol, to study in depth
the \openssl library, and to survey some of the attacks to which a bad key
generation is exposed. On the footprint of ~\cite{20years}, where most of these
attacks have already been analyzed, we are going to describe the mathematical
basis of each attack, reason about a possible solution in procedural
programming, and finally, give some hints about a distributed version of it.

Besides the pseudocode already available in this document, the project led to the
development of a real, open, C implementation consultable at
{\small \url{https://github.com/mmaker/bachelor}}.

In addition, the application has then been deployed on the
university cluster and pointed against a huge number of websites -
\href{http://www.alexa.com/}{Alexa}'s \emph{top 1 million global websites}.
Some of the statistical result extracted from this investigation are later
examined in chapter \ref{chap:empirical_study}.

%%% Local Variables:
%%% mode: latex
%%% TeX-master: "question_authority"
%%% End:
