\chapter{Dixon {\texttt{\small{[insert random buzzword here]}}} method\label{chap:dixon}}

~\cite{dixon} describes a class of ``probabilistic algorithms'' for finding a
factor of any composite number, at a sub-exponential cost. They basically
consists into taking random integers $r$ in $\{1, \ldots, N\}$ and look for those
where $r^2 \mod{N}$ is \emph{smooth}. If enough are found, then those integers
can somehow be assembled, and so a fatorization of N attemped.

%% that's not really academic to be stated officially, but I would have never
%% understood this section without Firas (thanks).
%% <http://blog.fkraiem.org/2013/12/08/factoring-integers-dixons-algorithm/>
%% I kept the voila` phrase, that was so lovely.
\section{A little bit of History}
During the latest century there has been a huge effort to approach the problem
formulated by Fermat ~\ref{eq:fermat_problem} from different perspecives. This
led to an entire family of algorithms known as \emph{Quadratic Sieve} [QS]. The
core idea is still to find a pair of perfect squares whose difference can
factorize $N$, but maybe Fermat's hypotesis can be made weaker.

\paragraph{Kraitchick} was the first one popularizing the idea the instead of
looking for integers $\angular{x, y}$ such that $x^2 -y^2 = N$ it is sufficient
to look for \emph{multiples} of $N$:
\begin{align}
  x^2 - y^2 \equiv 0 \pmod{N}
\end{align}
and, once found, claim that $\gcd(N, x \pm y)$ are non-trial divisors of $N$
just as we did in \ref{sec:fermat:implementation}.
Kraitchick did not stop here: instead of trying $x^2 \equiv y^2 \pmod{N}$ he
kept the value of previous attempt, and tries to find \emph{a product} of such
values which is also a square. So we have a sequence
\begin{align}
  \label{eq:dixon:x_sequence}
  \angular{x_0, \ldots, x_k} \mid \forall i \leq k \quad x_i^2 - N
  \; \text{ is a perfect square}
\end{align}
and hence
\begin{align*}
  \prod_i (x_i^2 - N) = y^2
\end{align*}
that $\mod{N}$ is equivalent to:
\begin{align}
  \label{eq:dixon:fermat_revisited}
  y^2 \equiv \prod_i x_i^2 - N \equiv \big( \prod_i x_i \big) ^2 \pmod{N}
\end{align}
and voil\`a our congruence of squares. For what concerns the generation of $x_i$
with the property \ref{eq:dixon:x_sequence}, they can simply taken at random and
tested using trial division.

\paragraph{Brillhart and Morrison} later proposed (\cite{morrison-brillhart}
p.187) a better approach than trial division to find such $x$. Their idea aims
to ease the enormous effort required by the trial division. In order to achieve
this. they introduce a \emph{factor base} $\factorBase$ and generate random $x$
such that $x^2 - N$ is $\factorBase$-smooth. Recalling what we anticipated in
~\ref{sec:preq:numbertheory}, $\factorBase$ is a precomputed set of primes
$p_i \in \naturalPrime$.
This way the complexity of generating a new $x$ is dominated by
\bigO{|\factorBase|}. Now that the right side of \ref{eq:dixon:fermat_revisited}
has been satisfied, we have to select a subset of those $x$ so that their
product can be seen as a square. Consider an \emph{exponent vector}
$v_i = (\alpha_0, \alpha_1, \ldots, \alpha_r)$ associated with each $x_i$, where
\begin{align*}
  a_j = \begin{cases}
    1 \quad \text{if $p_j$ divides $x_i$ to an odd power} \\
    0 \quad \text{otherwise}
    \end{cases}
\end{align*}
for each $0 \leq j \leq r $. There is no need to restrict ourselves for positive
values of $x^2 -N$, so we are going to use $\alpha_0$ to indicate the sign. This
benefit has a neglegible cost: we have to add the non-prime $-1$ to our factor
base.

Let now $\mathcal{M}$ be the rectangular matrix having per each $i$-th row the
$v_i$ associated to $x_i$: this way each element $m_{ij}$ will be $v_i$'s
$\alpha_j$. We are interested in finding set(s) of $x$ that satisfies
\ref{eq:dixon:fermat_revisited}, possibly all of them.
Define $K$ as the subsequence of $x_i$ whose product always have even powers.
This is equivalent to look for the set of vectors $\{ w \mid wM = 0 \}$ by
definition of matrix multiplication in $\mathbb{F}_2$.


\paragraph{Dixon} Morrison and Brillhart's ideas of \cite{morrison-brillhart}
were actually used for a slightly different factorization method, employing
continued fractions instead of the square difference polynomial. Dixon refined
those by porting to the quare problem, achieving a probabilistic factorization
method working at a computational cost asymptotically  best than all other ones
previously described: \bigO{\beta(\log N \log \log N)^{\rfrac{1}{2}}} for some
constant $\beta > 0$ \cite{dixon}.

\section{Computing the Kernel}

%%% Local Variables:
%%% mode: latex
%%% TeX-master: "question_authority"
%%% End:
