\chapter{The Secure Layer \label{chap:ssl}}

Transport Layer Security, formerly known as SSL (Secure Socket Layer), aims
to bring some security features over a communication channel, specifically
providing \strong{integrity} and \strong{confidentiality} of the message, \strong{authenticity} of the server and
optionally the client.
%% fuck osi layers: there is no code explicitly structuring the internet in 7
%% layers.
The most allocate TLS in the 6 or 7th OSI Layer, ``Application'', and is nowdays widely adopted
all over the world, being the de-facto standard for end-to-end  encryption.

\paragraph{Certifications Authority} are at the root of the security of the
protocol. See section ~\ref{sec:ssl:x509}

\paragraph{The protocol} is actually composed of many sub-protocols:

\begin{itemize}
\item handshake protocol
\item record protocol
\item alert protocol
\item changespec protocol ?
\end{itemize}
We will proceed by describing in deep only the first two of these, due to their
relevant role inside the conection and furthermore, because they are the only
two we actually made use of during our investigations.


\section{The \texttt{handshake} protocol}
Different options:
\begin{itemize}
\item no session
\item session
\item client authenticaton
\end{itemize}


\section{The \texttt{record} protocol}

Until 2005, failure to authenticate, decrypt will result in I/O error and a
close of the connection

\section{What's inside a certificate \label{sec:ssl:x509}}

\section{Remarks among SSL/TLS versions}

cos'e
differenze tra le varie versioni
la certification autority
%%% Local Variables:
%%% mode: latex
%%% TeX-master: "question_authority.tex"
%%% End:
