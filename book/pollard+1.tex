\chapter{Williams' $p+1$ factorization method \label{chap:william+1}}

Analogously to Pollard's $p-1$ factorization described in chapter
~\ref{chap:pollard-1}, this method will allow the determination of the divisor
$p$ of a number $N$, if $p$ is such that $p+1$ has only small prime divisors.
This method was presented in ~\cite{Williams:p+1} together with the results of
the application of this method to a large number of composite numbers.

\begin{remark}
  In the end of ~\cite{Williams:p+1}, there is a small performance comparison
  with Pollard's $p-1$:
  ``The real problem with the $p+1$ test is the fact that it is quite slow. For
  our program, we found that it was about nine times slower.''
  Nevertheless, there is no further information about the way the two
  factorization have been benchmarked.
\end{remark}
%%% Local Variables:
%%% mode: latex
%%% TeX-master: "question_authority"
%%% End: