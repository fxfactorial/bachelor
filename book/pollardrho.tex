\chapter{Pollard's $\rho$ factorization method \label{chap:pollardrho}}

A \emph{Monte Carlo} factorization method, published by J. M. Pollard in
~\cite{pollardMC}, consists into identifying a periodically recurrent  sequence
of integers within a random walk $\pmod{N}$ that could leak one of the two
factors.

Consider a function $f$ from $\mathcal{S}$ to $\mathcal{S}$, where
$\mathcal{S} = \{0, 1, \ldots, q-1\}$ and $q \in \naturalPrime$.
Let $s$ be a random element in $\mathcal{S}$, and consider the sequence
\begin{align*}
  s,\ f(s),\ f(f(s)),\ \ldots
\end{align*}
Since $f$ acts over a finite set, it is clear that this sequence must
eventually repeat, and become cyclic.
We might diagram it with the letter $\rho$, where the tail represent the
aperiodic part, or \emph{epacts}, and the oval the cyclic part, or
\emph{period}.
\begin{center}
  \begin{tikzpicture}[scale=0.7, thick]
    \tikzstyle{every node}=[draw,circle,fill=white,minimum size=4pt,
                            inner sep=0pt]
    \node (1) at (1.4, 0.2) [label=left:$x_1$] {};
    \node (2) at (2.5, 3)   [label=left:$x_{i-2}$] {};
    \node (3) at (3.25, 5)  [label=left:$x_{i-1}$] {};
    \node (4) at (4, 7)     [label=left:$ x_i \equiv x_j $] {};
    \node (5) at (6, 9)     [label=above:$x_{i+1}$] {};
    \node (6) at (8, 7)     [label=right:$x_{i+2}$] {};
    \node (7) at (6, 5)     [label=below:$x_{j-1}$] {};

    \path (1) edge [dashed] (2);
    \path (2) edge (3);
    \path (3) edge (4);
    \path (4) edge [bend left] (5);
    \path (5) edge [bend left] (6);
    \path (6) edge [bend left, dashed] (7);
    \path (7) edge [bend left] (4);

    %%\draw [decorate,decoration={brace, raise=1.5cm}] (1) -- (3)
    %%node[draw=no] at (-1.5, 4) {tail};
    \draw [decorate,decoration={brace, raise=3cm}] (5) -- (7)
    node[draw=none] at (13, 7) {\footnotesize {period $\pmod{q}$}};

\end{tikzpicture}
\end{center}

Now, consider $N=pq$.
Let $F(x)$ be any function generating pseudorandom integers
$\angular{x_1, x_2, \ldots}$, and let $f(x) = F(x) \pmod{q}$.
As we said above, without any luck, there will be a pair $\angular{x_i, x_j}$
generated by $F$ such that $x_i \equiv x_j \pmod{q}$, but $x_i \neq x_j$.

Therefore, in order to factorize $N$, we proceed as follows: starting from a
random $s$, we iteratively apply $F$ reduced modulo $N$. Whenever we find a
period, if $\gcd(x_i - x_j, N) > 1$ then we found a non-trivial
factor of $N$.

\paragraph{Choosing the function} Ideally, $F$ should be easily computable, but
at the same time random enough to reduce as much as possible the epacts
~\cite{Crandall} \S 5.2.1. Any quadratic function $F(x) = x^2 + b$ should be
enough\footnote{
  Note that this has been only empirically verified, and so far not been proved
  (~\cite{riesel}, p. 177)},
provided that $b \in \naturalN \setminus \{0, 2\}$.
For example, ~\cite{pollardMC} uses $x^2 -1$, meanwhile we are going to choose
$F(x) = x^2 + 1$.

\paragraph{Finding the period} The trivial way to discover a period would be to
test $x_i$ with all $x_j, \quad j < i$. Though, in \cite{AOCPv2} \S 3.1,
Knuth gives a simple and elegant algorithm, attributed to Floyd, for finding a
multiple of the period.
This algorithm is the same one finally adopted by Pollard in
~\cite{pollardMC}.

\begin{theorem*}[Floyd]
Given an \emph{ultimately periodic} sequence, in the sense that there exists
numbers $\lambda$ and $\mu$ for which the values:
\begin{align*}
  X_0, X_1, \ldots, X_{\mu}, \ldots, X_{\mu + \lambda - 1}
\end{align*}
are distinct, but $X_{n+\lambda} = X_n$ when $n \geq \mu$,
then there exists an
$\mu < n < \mu + \lambda$ such that $X_n = X_{2n}$.
\end{theorem*}

\begin{proof}
  First, if $X_n = X_{2n}$, then the sequence is obviously periodic from
  $X_{2n}$ onward, possibly even earlier.
  Conversely, it is true that $X_n = X_m \quad (n \geq \mu)$ for
  $m = n + k\lambda, \quad k \in \naturalN$. Hence, there will eventually
  be an $n$ such that $X_n = X_{2n}$ if and only if $n - \mu$ is a multiple of
  $\lambda$.
  The first such value happens for $n = (\lambda + 1)\floor{\rfrac{\mu}{\lambda}}$.
\end{proof}

The immediate consequence of this is that we can find a collision simply by
checking $\gcd(x_{2i} - x_i, N) > 1$ for incremental $i$.

\paragraph{Brent's Improvement} In 1979, Brent discovered an entire family of
cycle-finding algorithms whose optimal version resulted to be 36\% faster than
Floyd's one \cite{pollard-brent}.
Instead of looking for the period of the sequence using $x_{2i} - x_i$, Brent
considers
$\abs{x_j - x_{2^k}}$ for $ 3 \cdot 2^{k-1} < j \leq 2^{k+1}$, resulting in
fewer operations required by the algorithm. Pragmatically, this boils down to
compare:

\medskip
\begin{tabular}{l@{\hskip 40pt} l@{\hskip 50pt} l}
  $k = 0$ & $j \in \{1+1 \ldots 2\}$ & $\abs{x_1 - x_2}$ \\
  $k = 1$ & $j \in \{3+1 \ldots 4\}$ & $\abs{x_2 - x_4}$ \\
  $k = 2$ & $j \in \{6+1 \ldots 8\}$ & $\abs{x_4 - x_7}$, $\abs{x_4 - x_8}$ \\
  $k = 3$ & $j \in \{12+1, \ldots 16\}$ &
            $\abs{x_8 - x_{13}}$, $\abs{x_8 -x_{14}}$, $\ldots$, $|x_8 - x_{16}|$\\
  $k = 4$ & $j \in \{24+1, \ldots 32 \}$ &
            $\abs{x_{16} - x_{25}}$, $\ldots$, $\abs{x_{16} - x_{32}}$ \\[2pt]
  $\quad \vdots$ & $\quad \vdots$ & $\quad \quad \vdots$ \\
  %\multicolumn{1}{c}{$\vdots$} &
  %\multicolumn{1}{c}{$\vdots$} &
  %\multicolumn{1}{c}{$\vdots$} \\
\end{tabular}


A Pollard's $\rho$ variant that implements Brent's cycle-finding algorithm
instead of Floyd's one runs around 25\% faster on average
~\cite{pollard-brent}.

\section{Complexity}
\cite{riesel} presents a nice proof of the \emph{average} complexity of
this algorithm, based on the birthday paradox.
\newtheorem*{birthday}{The Birthday Paradox}
\begin{birthday}
  How many persons needs to be selected at random in order that the probability
  of at least two of them having the same birthday exceeds $\rfrac{1}{2}$?
\end{birthday}

\begin{proof}[Solution]
  The probability that $\epsilon$ different persons have different birthdays is:
  \begin{align*}
    \Big(1 - \frac{1}{365}\Big)
    \Big(1 - \frac{2}{365}\Big)
    \Big(1 - \frac{3}{365}\Big)
    \cdots
    \Big(1 - \frac{\epsilon -1}{365}\Big)
    =
    \frac{365!}{365^\epsilon (365-\epsilon)!}
  \end{align*}
  This expression becomes $< \rfrac{1}{2}$ for $\epsilon \geq 23$.
\end{proof}

We can obviously substitute the $365$ with any set of cardinality $\zeta$
to express the probability that a random function from $\integerZ_{\epsilon}$
to $\integerZ_{\zeta}$ is injective. However, back to our particular case,
we want to answer the question:

\emph{
  How many random numbers do we have to run through before finding at least
  two integers equivalent $\mod{q}$?
}


Using the same reasoning presented above over the previously defined function
$f(x): \mathcal{S} \to \mathcal{S}$, we will discover that after
$\approx 1.18 \sqrt{q}$ steps the probability to have fallen inside the period
is $\rfrac{1}{2}$. %% is it clear that q is either one of the two primes, and
                   %% that here we want to examinate only a portion of the
                   %% domain?
Since any of the two primes factoring $N$ is bounded above by $\sqrt{N}$, we
will find a periodic sequence, and thus a factor, in time \bigO{\sqrt[4]{N}}.


\section{A Computer program for Pollard's $\rho$ method}

The initial algorithm described by Pollard \cite{pollardMC} and consultable
immediately below, looks for the pair $\angular{x_i, x_{2i}}$ such that
$\gcd(x_{2i} - x_i, N) > 1$.  This is achieved by keeping two variables $x, y$
and respectively updating them via $x \gets f(x)$ and $y \gets f(f(y))$.

\begin{algorithm}
  \caption{Pollard's $\rho$ factorization}
  \begin{algorithmic}[1]
    \Function{rho}{\PKArg}
    \State $x \getsRandom \naturalN$
    \State $y \gets x$
    \State $g \gets 1$
    \While{$g = 1$}
      \State $x \gets x^2 + 1 \pmod{N}$
      \State $y \gets y^4 + 2y^2 + 2 \pmod{N}$
      \State $g \gets gcd(x, y)$
    \EndWhile
    \Return $g$
    \EndFunction
  \end{algorithmic}
\end{algorithm}

\begin{remark}
  It is intresting to see how in its basic version, Pollard's $\rho$
  method just needs 3 variables  to preserve the
  state. This places it among the most parsimonious factorization algorithms in
  terms of memory footprint.
\end{remark}

An immediate improvement of this algorithm would be to occasionally compute Euclid's
algorithm over the accumulated product to save some computation cycles, just as
we saw in section~\ref{sec:pollard-1:implementing}. The next code fragment
adopts this trick together with Brent's cycle-finding variant:

\begin{algorithm}
  \caption{Pollard-Brent's factorization \label{alg:pollardrho}}
  \begin{algorithmic}[1]
    \Function{rho}{\PKArg}
    \State $s \gets 100$
    \Comment steps to check for $\gcd$
    \State $i \gets 1; \quad j' \gets j \gets 1$
    \Comment step and period counters
    \State $x' \gets x \getsRandom \naturalN$
    \Comment the $x_i$ discussed above
    \State $y' \gets y \gets x^2 + 1$
    \Comment the $x_j$ discussed above
    \State $k \gets 0; \quad q \gets \abs{x-y}$
    \While{$g = 1$}
      \State $x \gets y$ \Comment $x_i = 2^k$
      \State $j \gets 3 \cdot 2^{k} + 1$
      \State $k \gets k+1$
      \While{$j \leq 2^k$}
        \State $j \gets j+1; \quad i \gets i+1$
        \Comment increment step and period counters
        \State $y \gets y^2 + 1 \pmod{N}$
        \State $q \gets q \cdot \abs{x - y}$
        \If{$ i \mid s$} \Comment time to compute $\gcd$?
          \State $g \gets \gcd(q, N)$
          \If{$g = N$}
            \Comment too far: fall back to latest epoch
            \State $s \gets 1; \quad  g \gets 1$
            \State $j \gets j'; \quad x \gets x'; \quad y \gets y'$
          \ElsIf{$g = 1$}
            \Comment save current state
            \State $x' \gets x; \quad y' \gets y$
            \State $j' \gets j$
          \Else
            \State $p \gets g$
            \State $q \gets N//p$
            \State \Return $p, q$
          \EndIf
        \EndIf
      \EndWhile
    \EndWhile
    \EndFunction
  \end{algorithmic}
\end{algorithm}

%%% Local Variables:
%%% mode: latex
%%% TeX-master: "question_authority"
%%% End:
